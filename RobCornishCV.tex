%%%%%%%%%%%%%%%%%%%%%%%%%%%%%%%%%%%%%%%%%
% Developer CV
% LaTeX Template
% Version 1.0 (28/1/19)
%
% This template originates from:
% http://www.LaTeXTemplates.com
%
% Authors:
% Jan Vorisek (jan@vorisek.me)
% Based on a template by Jan Küster (info@jankuester.com)
% Modified for LaTeX Templates by Vel (vel@LaTeXTemplates.com)
%
% License:
% The MIT License (see included LICENSE file)
%
%%%%%%%%%%%%%%%%%%%%%%%%%%%%%%%%%%%%%%%%%

%----------------------------------------------------------------------------------------
%    PACKAGES AND OTHER DOCUMENT CONFIGURATIONS
%----------------------------------------------------------------------------------------

\documentclass[9pt]{developercv} % Default font size, values from 8-12pt are recommended

\newcommand{\CPP}
{C\nolinebreak[4]\hspace{-.05em}\raisebox{.22ex}{++ }}

%----------------------------------------------------------------------------------------

\begin{document}

%----------------------------------------------------------------------------------------
%    TITLE AND CONTACT INFORMATION
%----------------------------------------------------------------------------------------

\begin{minipage}[t]{0.45\textwidth} % 45% of the page width for name
    \vspace{-\baselineskip} % Required for vertically aligning minipages
    
    % If your name is very short, use just one of the lines below
    % If your name is very long, reduce the font size or make the minipage wider and reduce the others proportionately
    \colorbox{white}{{\HUGE\textcolor{black}{\textbf{{Rob Cornish}}}}} % First name
    
    % \colorbox{white}{{\HUGE\textcolor{black}{\textbf{{Cornish}}}}} % Last name
    
    \vspace{6pt}
    
    % {\huge AI Scientist \& Entrepreneur} % Career or current job title
\end{minipage}
\begin{minipage}[t]{0.275\textwidth} % 27.5% of the page width for the first row of icons
    \vspace{-\baselineskip} % Required for vertically aligning minipages
    
    % The first parameter is the FontAwesome icon name, the second is the box size and the third is the text
    % Other icons can be found by referring to fontawesome.pdf (supplied with the template) and using the word after \fa in the command for the icon you want
    % \icon{MapMarker}{12}{Black Mesa East}\\
    \icon{Globe}{12}{\href{https://jrmcornish.github.io}{jrmcornish.github.io}}\\
    \icon{At}{12}{\href{mailto:jrmcornish@gmail.com}{jrmcornish@gmail.com}}\\    
    % \icon{MortarBoard}{12}{\href{https://scholar.google.com/citations?user=ZDVQRN0AAAAJ}{Google Scholar}}%\\
\end{minipage}
\begin{minipage}[t]{0.275\textwidth} % 27.5% of the page width for the second row of icons
    \vspace{-\baselineskip} % Required for vertically aligning minipages
    
    % The first parameter is the FontAwesome icon name, the second is the box size and the third is the text
    % Other icons can be found by referring to fontawesome.pdf (supplied with the template) and using the word after \fa in the command for the icon you want
    \icon{Github}{12}{\href{https://github.com/jrmcornish}{github.com/jrmcornish}}\\
    \icon{Phone}{12}{+44 7908 000873}\\
    % \icon{Twitter}{12}{\href{https://twitter.com/@alyxvance}{@alyxvance}}\\
\end{minipage}

\vspace{1cm}

%----------------------------------------------------------------------------------------
%    INTRODUCTION, SKILLS AND TECHNOLOGIES
%----------------------------------------------------------------------------------------

% \cvsect{About me}

% \begin{minipage}[t]{\textwidth} % 40% of the page width for the introduction text
%     \vspace{-\baselineskip} % Required for vertically aligning minipages
%     {
%         TODO
%       I have a background in machine learning, mathematics and statistics, and electrical engineering, and am currently employed as a postdoctoral researcher in the Department of Statistics at the University of Oxford.
%       I wish to pursue a career in science, applying mathematical and statistical techniques rigorously to solve interesting problems and to impact the world in a postive way.
%       % I am particularly drawn to CNRS given the significant intellectual freedom it offers, and the open-minded culture of free inquiry that it provides.

%       % working on interesting problems for which a principled mathematical analysis 

%       % academia, which appeals to me due to the intellectual freedom 
      
%       % where I believe I will have the freedom to work on interesting 
      
%       % a DPhil at the University of Oxford working on techniques for properly quantifying uncertainty in very large-scale machine learning models.
%     }
%     % {I believe in the potential for technology to bring about positive large-scale societal change, and have dedicated myself to the pursuit of technical excellence in order to contribute to this.
%     % I have a background in machine learning, mathematics and statistics, and electrical engineering, and recently completed a DPhil at the University of Oxford working on techniques for properly quantifying uncertainty in very large-scale machine learning models.
%     % I am a co-founder of Quro Medical, a digital health startup pioneering technology-enabled hospital-at-home services in Southern Africa.
%     % Quro has provided me with an incredibly rewarding opportunity to apply my technical knowledge in a context where the potential for positive impact is immediate.
%     % The COVID-19 pandemic in particular has seen us rapidly scale up our offering to respond to the increased demand for high-quality remote care and monitoring services in order to relieve pressure on hospitals.
%     % Alongside this, I have been working as a postdoctoral researcher in the Department of Statistics at the University of Oxford, investigating machine learning techniques for healthcare applications that I hope to apply to the work we are doing at Quro Medical in the near future.}

% %   {\bf MODIFY}
%   % I completed a DPhil in machine learning at the University of Oxford as part of the AIMS CDT, with a focus on Bayesian statistics and deep learning. I hold first-class honours degrees in electrical engineering and in pure mathematics from the University of Melbourne, and in applied mathematics from Monash University. In all these I received various prizes for academic excellence. I am a passionate entrepreneur who believes in the potential of technology to bring about positive societal changes at a large scale.

%     % \lorem \lorem \lorem \lorem \lorem\\ % Dummy text
% \end{minipage}

% \vspace{.5cm}

% \hfill % Whitespace between
% \begin{minipage}[t]{0.5\textwidth} % 50% of the page for the skills bar chart
%     \vspace{-\baselineskip} % Required for vertically aligning minipages
%     \begin{barchart}{5.5}
%         \baritem{JavaScript}{60}
%         \baritem{PHP}{100}
%         \baritem{SASS/LESS}{70}
%         \baritem{Bootstrap}{70}
%         \baritem{Git}{40}
%         \baritem{LaTeX}{60}
%     \end{barchart}
% \end{minipage}

% \begin{center}
%     \bubbles{5/Eclipse, 6/git, 4/Office, 3/Inkscape, 3/Blender}
% \end{center}

%----------------------------------------------------------------------------------------
%    EXPERIENCE
%----------------------------------------------------------------------------------------

% \vspace{-4em}

\cvsect{Education}

\begin{entrylist}
  \entry
    {2015 -- 2020}
    {Doctor of Philosophy in Autonomous Intelligent Machines and Systems (AIMS CDT)}
    {University of Oxford}
    {\textit{Thesis title:} Algorithms and Modelling for Large-Scale Bayesian Data Analysis. \\
    \textit{Supervisors:} Arnaud Doucet, George Deligiannidis, Hongseok Yang, Frank Wood \\
    % \textit{College:} Balliol \\
    \textit{Funding:} Fully provided by the CDT
    }
  \entry
    {2014 -- 2015}
    {Bachelor of Science (Honours year)}
    {Monash University}
    {Major in Applied Mathematics \\
    \textit{Thesis title:} Detection and Tracking of Surface Primitives in RGBD Data\\
    \textit{Supervisors:} Tom Drummond and Norman Do}
  \entry
    {2012 -- 2014}
    {Bachelor of Science}
    {University of Melbourne}
    {Major in Electrical Systems}
  \entry
    {2011 -- 2014}
    {Diploma in Mathematical Sciences }
    {University of Melbourne}
    {Equivalent to a full major in Pure Mathematics within the Bachelor of Science}
  \entry
    {2010 -- 2012}
    {Bachelor of Arts}
    {University of Melbourne}
    {Double major in Philosophy, and History and Philosophy of Science (transferred to BSc in 2012)}
\end{entrylist}

\cvsect{Academic positions}
\begin{entrylist}
    \entry
    {Oct.\ 2025 --\\ present}
    {Assistant Professor (tenure track)}
    {Nanyang Technological University}
    {To commence in the College of Computing and Data Science (CCDS) in October 2025.}
    \entry
    {Sep.\ 2022 --\\ Sep.\ 2025}
    {Florence Nightingale Bicentenary Research Fellow}
    {Department of Statistics, University of Oxford}
    {An independent, three year, research-focussed position, with some teaching and admin responsibilities. Roughly equivalent to US non-tenure track Assistant Professor.}
    \entry
    {Sep.\ 2020 --\\ Aug.\ 2022}
    {Postdoctoral researcher}
    {Department of Statistics, University of Oxford}
    {
    \textit{Supervisors:} Arnaud Doucet and Chris Holmes}

\end{entrylist}

\cvsect{Grants}

\begin{entrylist}
    \entry
    {2025}
    {NRF-AI Fellowship ("In-principle Approved for Award")}
    {National Research Foundation (NRF), Singapore}
    {A 5 year early-career fellowship to support building a research group to work on groundbreaking, high-risk AI research in Singapore. I am currently waiting for final approval of my budget and milestones from the NRF. The total value of the Fellowship is up to SGD\$3.25 million over 5 years.}
\end{entrylist}

\cvsect{Entrepreneurship}

\begin{entrylist}
    \entry
        {2018 -- Present}%\\\footnotesize{part time}}
        {Cofounder, director (2018-present), \& CTO/CSO (2018-2021)}
        {Quro Medical}
        {Quro Medical is a South African digital health startup pioneering technology-enabled remote patient monitoring and hospital-at-home services in Southern Africa.
        Quro has successfully treated large numbers of patients with a variety of medical conditions, including many with COVID-19 during the pandemic, and successfully raised multiple rounds of funding.
        This includes rounds totalling \$1.1M USD in 2021, \$1.4M USD in 2023, and an additional undisclosed round in 2022 from Life Healthcare, South Africa’s second largest private hospital group.
        See \underline{\href{https://www.quromedical.co.za}{www.quromedical.co.za}} or our feature on \underline{\href{https://techcrunch.com/2021/04/14/south-africas-quro-medical-comes-out-of-stealth-with-1-1m-to-expand-its-hospital-at-home-service/}{TechCrunch}}.

        \vspace{1em}
        I was the only technical cofounder.
        Some of my contributions included: building and maintaining a minimum viable product of the platform, including Android, cloud, and front-end components (much of which is still in use today); hiring and managing a team of developers; and convincing potential investors and other stakeholders of the technical viability and promise of our overall strategy.}
\end{entrylist}


\cvsect{Patents}
\begin{entrylist}
  \entry
    {2023}
    {``Method of policy evaluation for contextual bandits'' (pending)}
    {European Patent Office}
    {ByteDance and the University of Oxford applied for a patent based on our paper ``Marginal Density Ratio for Off-Policy Evaluation in Contextual Bandits'. I am listed as a co-inventor. The application is currently pending.}
\end{entrylist}
	
\newpage
 
\cvsect{Awards and Achievements}
\begin{entrylist}
  \entry
    {2021}
    {Outstanding Reviewer award}
    {NeurIPS 2021}
    {Granted to top 10\% of reviewers}
  \entry
    {2020}
    {Offered place in Entrepreneur First (EF) London cohort for autumn 2020}
    {Entrepreneur First}
    {Declined in order to continue working on Quro Medical}
  \entry
    {2018}
    {Oxford VIEW Level 1 Best Idea \& Presentation Award}
    {Entrepreneurship Centre, Sa\"id Business School}
    {The Venture Ideation Exploration Workshop (VIEW) is a six-week course that focuses on exploring an entrepreneurial idea, and testing to see if there is a potentially feasible market for it.}
  \entry
    {2016}
    {Carl Moppert Prize in Mathematics}
    {School of Mathematical Sciences, Monash Univeristy}
    {Awarded for ``best overall Honours student in the school''}
  \entry
    {2014}
    {Mathematical Sciences Honours Scholarship}
    {School of Mathematical Sciences, Monash University}
    {Awarded to students who have received at least 90 per cent in 24 points of relevant level 3 mathematics units}
  \entry
    {2014}
    {John Monash Exhibition Prize}
    {Department of Electrical and Electronic Engineering, University of Melbourne}
    {Awarded for ``best overall performance in third-year undergraduate Electrical Engineering subjects''}
%   \entry
%     {2013}
%     {Winner of SumoRobo robotics competition}
%     {Melbourne University Electrical Engineering Club (MUEEC)}
%     {Along with two team members, designed and built the winning entry of a robot-sumo competition}
  \entry
    {2012}
    {Dean's Honours List for 2011}
    {Faculty of Arts, University of Melbourne}
    {Awarded to high achieving students in the Faculty of Arts}
  \entry
    {2011}
    {Hastie Exhibition}
    {School of Historical and Philosophical Studies, University of Melbourne}
    {Awarded to two undergraduate students with ``highest overall results in two Philosophy subjects''}
\end{entrylist}


\cvsect{Papers}

\begin{entrylist}
    \publication
        {Eilenberg-Moore Categories of Markov Monads}
        {\underline{Rob Cornish}}
        {8th International Conference on Applied Category Theory (extended abstract track)}
        {2025}

    \publication
        {SymDiff: Equivariant Diffusion via Stochastic Symmetrisation}
        {Leo Zhang, Kianoosh Ashouritaklimi, Yee Whye Teh, \underline{Rob Cornish}}
        {International Conference on Learning Representations (ICLR), 2025}
        {2025}

    \publication
        {Neural Network Symmetrisation in Concrete Settings}
        {\underline{Rob Cornish}}
        {Symmetry and Geometry in Neural Representations (NeurReps) Workshop, NeurIPS}
        {2024}

    \publication
        {Stochastic Neural Network Symmetrisation in Markov Categories}
        {\underline{Rob Cornish}}
        {Preprint (under review at JMLR)}
        {2024}

    \publication
        {Marginal Density Ratio for Off-Policy Evaluation in Contextual Bandits}
        {Muhammad Faaiz Taufiq, Arnaud Doucet, \underline{Rob Cornish}, Jean-Francois Ton}
        {Neural Information Processing Systems (NeurIPS), 2023}
        {2023}

    \publication
        {Causal Falsification of Digital Twins}
        {\underline{Rob Cornish}, Muhammad Faaiz Taufiq, Arnaud Doucet, Chris Holmes}
        {Preprint (under review at JMLR)}
        {2023}

    \publication
        {Conformal Off-Policy Prediction in Contextual Bandits}
        {Muhammad Faaiz Taufiq, Jean-Francois Ton, \underline{Rob Cornish}, Yee Whye Teh, Arnaud Doucet}
        {Neural Information Processing Systems (NeurIPS), 2022}
        {2022}

    \publication
        {Variational Inference with Continuously-Indexed Normalizing Flows}
        {Anthony L. Caterini, \underline{Rob Cornish}, Dino Sejdinovic, Arnaud Doucet}
        {Conference on Uncertainty in Artificial Intelligence (UAI), 2021}
        {2021}

    \publication
        {On Multilevel Monte Carlo Unbiased Gradient Estimation for Deep Latent Variable Models}
        {Yuyang Shi, \underline{Rob Cornish}}
        {International Conference on Artificial Intelligence and Statistics (AISTATS), 2021}
        {2021}

    % \publication
    %     {Deep Generative Missingness Pattern-Set Mixture Models}
    %     {Sahra Ghalebikesabi, \underline{Rob Cornish}, Chris Holmes, Luke Kelly}
    %     {International Conference on Artificial Intelligence and Statistics (AISTATS), 2021}
    %     {2021}

    \publication
        {Relaxing Bijectivity Constraints with Continuously Indexed Normalising Flows}
        {\underline{Rob Cornish}, Anthony L. Caterini, George Deligiannidis, Arnaud Doucet}
        {International Conference on Machine Learning (ICML), 2020}
        {2020}

    \publication
        {Robust Out-of-Sample Uncertainty for Neural Networks via Confidence Densities}
        {\underline{Rob Cornish}, George Deligiannidis, Arnaud Doucet}
        {ICML workshop on Uncertainty and Robustness in Deep Learning, 2019}
        {2019}

    \publication
        {Scalable Metropolis-Hastings for Exact Bayesian Inference with Large Datasets}
        {\underline{Rob Cornish}, Paul Vanetti, Alexandre Bouchard-C\^ot\'e, George Deligiannidis, Arnaud Doucet}
        {International Conference on Machine Learning (ICML), 2019}
        {2019}

    \publication
        {Online Learning Rate Adaptation with Hypergradient Descent}
        {At\i l\i m G\"une\c s Baydin, \underline{Rob Cornish}, David Rubio, Mark Schmidt, Frank Wood}
        {International Conference on Learning Representations (ICLR), 2018}
        {2018}

    \publication
        {On Nesting Monte Carlo Estimators}
        {Tom Rainforth, \underline{Rob Cornish}, Hongseok Yang, Andrew Warrington, Frank Wood}
        {International Conference on Machine Learning (ICML), 2018}
        {2018}

    \publication
        {Towards a testable notion of generalisation for generative adversarial networks}
        {\underline{Rob Cornish}, Frank Wood, Hongseok Yang}
        {NIPS workshop on Deep Learning: Bridging Theory and Practice, 2017}
        {2017}

    \publication
        {Efficient exact inference in discrete Anglican programs}
        {\underline{Rob Cornish}, Frank Wood, Hongseok Yang}
        {POPL workshop on Probabilistic Programming Semantics (PPS), 2017}
        {2017}

    \publication
        {Analyzing Array Manipulating Programs by Program Transformation}
        {\underline{Rob Cornish}, Graeme Gange, Jorge Navas, Peter Schachte, Harald S\o ndergaard, Peter Stuckey}
        {Logic-Based Program Synthesis and Transformation (LOPSTR): Proceedings of the 24th International Symposium}
        {2015}
\end{entrylist}

\cvsect{Invited talks}

\begin{entrylist}
    \entry
        {June 2025 (upcoming)}
        {Stochastic Neural Network Symmetrisation in Markov Categories}
        {Quantinuum}
        {\textit{Compositoinal Intelligence team seminar series}}
    \entry
        {May 2025}
        {Stochastic Neural Network Symmetrisation in Markov Categories}
        {University of Oxford}
        {\textit{Applied Category Theory (guest lecture for the MSc in Mathematics and Foundations of Computer Science)}}
    \entry
        {Feb.\ 2025}
        {An introduction to groups, actions, and equivariance}
        {University of Oxford}
        {\textit{Intelligent Earth CDT (guest lecture)}}
    \entry
        {Nov.\ 2024}
        {Stochastic Neural Network Symmetrisation in Markov Categories}
        {MIT}
        {\textit{Applied Category Theory for Engineering Design (ACT4ED) course (guest lecture)}}
    \entry
        {Oct.\ 2024}
        {Stochastic Neural Network Symmetrisation in Markov Categories}
        {Univeristy of Oxford}
        {\textit{RainML Seminar}}
    \entry
        {Oct.\ 2024}
        {Stochastic Neural Network Symmetrisation in Markov Categories}
        {University College London}
        {\textit{UCL Statistical Science Seminar}}
    \entry
        {Jul.\ 2024}
        {Stochastic Neural Network Symmetrisation in Markov Categories}
        {Nanyang Technological University}
        {\textit{CCDS Seminar series}}
    \entry
        {Jul.\ 2024}
        {Stochastic Neural Network Symmetrisation in Markov Categories}
        {University of Oxford}
        {\textit{DeepProb reading group (led by Yee Whye Teh)}}
    \entry
        {Jun.\ 2024}
        {Equivariant Stochastic Neural Networks in Markov Categories}
        {University of Oxford}
        {\textit{Applied Category Theory (ACT) Conference, 2024}}
    \entry
        {Apr.\ 2024}
        {Equivariant Stochastic Neural Networks in Markov Categories}
        {Amazon Berlin}
        {\textit{StatML CDT workshop}}
    \entry
        {Mar.\ 2024}
        {An Introduction to Categorical Probability Theory}
        {Imperial College London}
        {\textit{Statistics seminar series}}
    \entry
        {Feb.\ 2023}
        {An Introduction to Categorical Probability Theory}
        {Online}
        {\textit{CoSInES x Bayes4Health Seminar series}}
    \entry
        {Jan.\ 2023}
        {Causal Falsification of Digital Twins}
        {London School of Economics}
        {\textit{Joint Econometrics and Statistics Seminar Series}}
    \entry
        {Nov.\ 2022}
        {Normalising Flows and Continuously-Indexed Flows for Machine Learning}
        {University of Cambridge}
        {\textit{CoSInES-Bayes4Health Masterclass on Variational Inference}}
    \entry
        {Oct.\ 2022}
        {Causal Falsification of Digital Twins}
        {University of Oxford}
        {\textit{Oxford-GSK Institute of Molecular and Computational Medicine (IMCM) meeting}}
    \entry
        {Jun.\ 2022}
        {An Introduction to Categorical Probability Theory}
        {University of Oxford}
        {\textit{D\textasciicircum 2 reading group}}
    \entry
        {Apr.\ 2022}
        {Causal Falsification of Digital Twins}
        {University of Cambridge}
        {\textit{CSML reading group}}
    \entry
        {Dec.\ 2021}
        {Normalising Flows and Continuously-Indexed Flows for Machine Learning}
        {University of Cambridge}
        {\textit{MRC Biostatistics Unit Reading Group}}
    \entry
        {Sep.\ 2021}%\\\footnotesize{part time}}
        {Variational Inference with Continuously Indexed Normalising Flows}
        {University of Cambridge}
        {\textit{Bayes4Health Workshop}}
\end{entrylist}

\cvsect{Research supervision}

\begin{entrylist}
    \entry{Nov.\ 2025 -- present}
        {Undergraduate thesis supervisor (Part C/OMMS)}
        {Mathematical Institute, University of Oxford}
        {\emph{Students:} Faris Saadat-Yazdi, Edward Clarke, Paul Francis}
    \entry
        {2024 -- present}
        {DPhil supervisor}
        {Department of Statistics, University of Oxford}
        {\emph{Student:} Leo Zhang}
    \entry
        {2024\\\footnotesize{summer}}
        {MSc supervisor}
        {Department of Statistics, University of Oxford}
        {\emph{Student:} Peter Matthews \\ \emph{Thesis title:} Computing Inverse Models for Bayesian Inference Using String Diagrams}
    \entry
        {2024\\\footnotesize{summer}}
        {MSc supervisor}
        {Department of Statistics, University of Oxford}
        {\emph{Student:} Shufan Yang \\ \emph{Thesis title:} Assessing Strategies for Parameterising Permutation-equivariant Functions}
    \entry
        {2023\\\footnotesize{summer}}
        {MSc supervisor}
        {Department of Statistics, University of Oxford}
        {\emph{Student:} Conor Bateman \\ \emph{Thesis title:} Strategies for Confidence Prediction without Exchangeability}
    \entry
        {2022\\\footnotesize{summer}}
        {MSc supervisor}
        {Department of Statistics, University of Oxford}
        {\emph{Student:} Roel Hulsman \\ \emph{Thesis title:} Distribution-Free Finite-Sample Guarantees and Split Conformal Prediction}
    \entry
        {2021\\\footnotesize{summer}}
        {MSc supervisor}
        {Department of Statistics, University of Oxford}
        {\emph{Student:} Ziyi Yan \\ \emph{Thesis title:} A Comparison of Methods for Off-Policy Evaluation under Unobserved Confounding}
        % {Supervised the Statistics Masters student Ziyi Yan, which included formulating her project topic (``A Comparison of Methods for Off-Policy Evaluation under Unobserved Confounding''), meeting regularly to ensure she was on track, providing feedback on drafts, and marking her submission along with an external examiner}
    \entry
        {2020 -- 2024}
        {DPhil supervisor}
        {Department of Statistics, University of Oxford}
        {\emph{Student:} Muhammad Faaiz Taufiq (cosupervised by Arnaud Doucet and Yee Whye Teh) \\
        \emph{Thesis title:} Uncertainty Quantification and Causal Considerations for Off-Policy Decision Making}
\end{entrylist}

\cvsect{Teaching and service}

\begin{entrylist}
    \entry
        {Jan.\ 2024}
        {Confirmation examiner (along with Aleks Kissinger)}
        {Department of Computer Science, University of Oxford}
        {\emph{Student:} Ned Summers (supervised by Sam Staton) \\
        %\emph{Co-examiner:} Aleks Kissinger \\
        \emph{Report title:} Symmetry Theorems as Limits in Categories of Probabilistic Processes}

    \entry
        {Sep.\ 2023}
        {StatML CDT mini-course on Conformal Prediction}
        {Department of Statistics, University of Oxford}
        {Co-organised and taught along with Edwin Fong (faculty at HKU). We gave an introduction to the topic for complete beginners, and then covered some of our recent research in this area. We also had a guest lecture from Vladimir Vovk, as well as from researchers at Berkeley, MIT, Google Research, DeepMind, and Amazon.}

    \entry
        {2023-2025\\\footnotesize{Hilary term}}
        {Lecturer}
        {Department of Statistics, University of Oxford}
        {Taught half of ``Computational Statistics'', a 2\textsuperscript{nd}-year undergraduate \& MSc course with \textasciitilde120 students total}

    \entry
        {2014 -- 2015}%\\\footnotesize{part time}}
        {Mathematics tutor}
        {School of Mathematical Sciences, Monash University}
        {Taught calculus and linear algebra to classrooms of 20-30 undergraduate students over two semesters}
\end{entrylist}

\cvsect{Technical skills}

% Even within the context of academic research,
I believe it is important to be as familiar as possible with low-level, practical details of current and emerging technologies. %, since there are many problems that only become visible at this level.
% I have a proven track record of quickly adapting to new technologies at all layers of abstraction in order to achieve the requirements of a given project.
To this end, I have completed large-scale projects involving each of the following:

\vspace{.5em}


%----------------------------------------------------------------------------------------
%    ADDITIONAL INFORMATION
%----------------------------------------------------------------------------------------

\begin{minipage}[t]{0.3\textwidth}
    \vspace{-\baselineskip} % Required for vertically aligning minipages

    \cvsubsect{Programming languages}

    Python\\
    Kotlin\\
    Typescript\\
    Ruby\\
    Bash\\
    % C\\
    \CPP \\
    Clojure
    % \textbf{English} - native\\
    % \textbf{German} - proficient\\
    % \textbf{Polish} - rudimentary
\end{minipage}
\hfill
\begin{minipage}[t]{0.3\textwidth}
    \vspace{-\baselineskip} % Required for vertically aligning minipages
    
    \cvsubsect{Frameworks}
    
    PyTorch \\
    Tensorflow \\
    Pyro (and other PPLs) \\
    AWS CDK \\
    React \\
    LLVM \\ 
    Robot Operating System (ROS)
\end{minipage}
\hfill
\begin{minipage}[t]{0.3\textwidth}
    \vspace{-\baselineskip} % Required for vertically aligning minipages
    
    \cvsubsect{Platforms}
    
    AWS \\
    Android \\
    Linux \\
    Arduino
\end{minipage}

\vspace{.25cm}

%----------------------------------------------------------------------------------------



\end{document}
